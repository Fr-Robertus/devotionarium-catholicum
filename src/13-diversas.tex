% chktex-file 36
\newpage
\begin{center}
    \textbf{Orações Diversas}
\end{center}
\begin{center}
    Oração antes de começar a estudar \\ \textcolor{VioletRed3}{\scriptsize{(Oração de Santo Tomás de Aquino)}}
\end{center}
\begin{flushleft}
    Infalível Criador, que dos tesouros da vossa sabedoria, tiraste as hierarquias doa Anjos colocando-as com ordem admirável no céu e distribuístes o universo com encantável harmonia, Vós que sois a verdadeira fonte de luz e o princípio supremo da sabedoria, difundi sobre as trevas da minha mente o raio do esplendor, removendo as duplas trevas nas quais eu nasci: o pecado e a ignorância. Vós que tornaste fecunda a língua das crianças, tornai erudita a minha língua e espalhai sobre os meus lábios a vossa bênção. Concedei-me o discernimento para entender, a capacidade de reter, a sutileza revelar, a facilidade de aprender, a graça abundante de falar e de escrever. Ensinai-me a começar, regei-me a continuar e perseverar até o término. Vós que sois verdadeiro Deus e verdadeiro homem, e que viveis e reinais pelos séculos dos séculos. Amém.
\end{flushleft}
\begin{center}
    Oração pela aceitação da morte
\end{center}
\begin{flushleft}
    Meu Deus e meu Pai, Senhor da vida e da morte, que para justo castigo das nossas culpas, com um decreto imutável determinastes que todos os homens haviam de morrer, olhai para mim prostrado diante de Vós. Detesto de todo o coração as minhas culpas passadas, pelas quais mereci mil vezes a morte, que aceito agora como o fim de expiá-las e para obedecer à vossa amável vontade. De bom grado morrerei, Senhor, no momento, no lugar e do modo que Vós quiserdes, e aproveitarei até esse instante os dias que me restem de vida para lutar contra os meus defeitos e aumentar o meu amor por Vós, para quebrar os laços que atam o meu coração às criaturas e preparar a minha alma para comparecer à vossa presença; e desde agora me abandono sem reservas nos braços da vossa paternal Providência.
\end{flushleft}
\newpage
\begin{center}
    Oração a São José para santificar o trabalho
\end{center}
\begin{flushleft}
    Ó glorioso São José, modelo de todos os que se consagram ao trabalho, alcançai-me a graça de trabalhar com espírito de penitência, em expiação dos meus pecados; de trabalhar com consciência, pondo o cumprimento do meu dever acima das minha inclinações naturais; de trabalhar com agradecimento e alegria, olhando como uma honra o poder desenvolver por meio do trabalho os dons recebidos de Deus. Alcançai-me a graça de trabalhar com ordem, constância, intensidade e presença de Deus, sem jamais retroceder ante as dificuldades; de trabalhar, acima de tudo, com pureza de intenção e desapego de mim mesmo, tendo sempre diante dos olhos todas as almas e as contas que prestarei a Deus: a do tempo perdido, das habilidades inutilizadas, do bem omitido e das vaidades estéreis em meus trabalhos, tão contrários à obra de Deus. Tudo por Jesus, tudo por Maria, tudo à vossa imitação, ó patriarca São José. Esse será o meu lema na vida e na hora da morte. Amém.
\end{flushleft}
\begin{center}
    Oração para alcançar a virtude da pureza
\end{center}
\begin{flushleft}
    Glorioso São José, pai e protetor das virgens, guarda fiel a quem Deus confiou Jesus Cristo, a perfeita inocência, e Maria, a Virgem das virgens. Eu vos peço por Jesus e Maria, esse duplo tesouro a vós tão caro. Com vosso auxílio, dai-me conservar meu corpo isento de toda mancha, e que puro e casto sirva perpetuamente a Jesus e Maria em perfeita castidade. Amém.
\end{flushleft}
\begin{center}
    Oração para obter vitória contra as tentações
\end{center}
\begin{flushleft}
    Meu Deus, \textit{não me lanceis da vossa presença} (\textcolor{VioletRed3}{Sl 50,13}). Sei perfeitamente que não me abandonareis nunca, se for eu o primeiro a vos abandonar, ai! o que me faz temer esta desgraça é a experiência que tenho na minha fraqueza. Senhor, a vós pertence dar-me a força que hei mister contra o Inferno, que pretende reduzir-me ainda sob a sua escravidão; pelo amor de Jesus Cristo vo-la peço. Ó meu Salvador, estabelecei comigo uma paz perpétua, uma união eternamente indissolúvel. A este efeito, dai-me o vosso santo amor.\ \textit{Aquele que não vos ama é morto} (\textcolor{VioletRed3}{1Jo 3, 14}); a vós toca livrar-me desta desgraçada morte, ó Deus da minha alma! Ah! Pela amarga morte que por mim sofrestes, não permitais, a vós suplico, meu Jesus, consinta eu em perder a vossa amizade. Amo-vos sobre todas as coisas; espero permanecer sempre nos laços do vosso santo amor; neles morrer um dia, e neles viver eternamente. Ó Maria, sois a Mãe e dispensadora da perseverança, de vós é portanto que exijo e espero este grande dom. Amém.
\end{flushleft}
\newpage
\begin{center}
    Oração pelos defuntos \\
    \hfill{} \break{}
    \textcolor{VioletRed3}{Responso}
\end{center}
\begin{flushleft}
    Eu sou a ressurreição e a vida; quem crê em Mim, mesmo que esteja morto, viverá; e quem vive e crê em Mim não morrerá eternamente (\textcolor{VioletRed3}{Jo 11, 25}).
    \vspace{.2cm} \\
    Santos de Deus, vinde em seu auxílio; anjos do Senhor, correi ao seu encontro! Acolhei a(s) sua(s) alma(s), levando-a(s) à presença do Altíssimo.
    \vspace{.2cm} \\
    \VbarRed{} Cristo te(vos) chamou. Ele te(vos) receba, e os anjos te(vos) acompanhem ao seio de Abraão. \\
    \RbarRed{} Acolhei a(s) sua(s) alma(s), levando-a(s) à presença do Altíssimo.
    \vspace{.2cm} \\
    \VbarRed{} Dai-lhe(s), Senhor, o repouso eterno e brilhe para ele(s) a vossa luz. \\
    \RbarRed{} Acolhei a(s) sua(s) alma(s), levando-a(s) à presença do Altíssimo.
    \vspace{.2cm} \\
    \VbarRed{} Senhor, tende piedade de nós. \\
    \RbarRed{} Cristo, tende piedade de nós. \\
    \VbarRed{} Senhor, tende piedade de nós.
    \vspace{.2cm} \\
    Pai nosso.
    \vspace{.2cm} \\
    \VbarRed{} Descanse(m) em paz. \\
    \RbarRed{} Amém.
    \vspace{.2cm} \\
    Oração
    \vspace{.2cm} \\
    Ouvi ó Pai, as nossas preces; sede misericordioso para com o(s) vosso(s) servo(s) \textcolor{VioletRed3}{N.}, que chamastes deste mundo. Concedei-lhe (s) a luz e a paz no convívio dos vossos santos. Por Nosso Senhor Jesus Cristo, na unidade do Espírito Santo. \\
    \RbarRed{} Amém.
    \vspace{.2cm} \\
    Oração
    \vspace{.2cm} \\
    Absolvei, Senhor, a(s) alma(s) do(s) vosso(s) servo(s) \textcolor{VioletRed3}{N.} de todos os laços do pecado, a fim de que, na ressurreição gloriosa, entre os vossos Santos e eleitos, possa(m) ele(s), ressuscitado(s) em seu(s) corpo(s), de novo respirar(em). Por Cristo Nosso Senhor. \\
    \RbarRed{} Amém.
    \vspace{.2cm} \\
    \VbarRed{} Eu sou a ressurreição e a vida; quem crê em Mim, mesmo que esteja morto, viverá; e quem vive e crê em Mim não morrerá eternamente. Dai-lhe(s), Senhor, o repouso eterno. \\
    \RbarRed{} E brilhe(m) para ele(s) a vossa luz.
    \vspace{.2cm} \\
    \VbarRed{} Descanse(m) em paz. \\
    \RbarRed{} Amém.
    \vspace{.2cm} \\
    \VbarRed{} A(s) sua(s) alma(s) e as almas de todos os fiéis defuntos, pela misericórdia de Deus, descansem em paz. \\
    \RbarRed{} Amém.
\end{flushleft}
\begin{center}
    Oração de Fátima
\end{center}
\begin{flushleft}
    Meu Deus, eu creio, adoro, espero e amo-Vos. Peço-Vos perdão para os que não creem, não adoram, não esperam e não Vos amam. Santíssima Trindade, Pai, Filho, Espírito Santo, adoro-Vos profundamente e ofereço-Vos o preciosíssimo Corpo, Sangue, Alma e Divindade de Jesus Cristo, presente em todos os sacrários da terra, em reparação dos ultrajes, sacrilégios e indiferenças com que Ele mesmo é ofendido. E pelos méritos infinitos do Seu Santíssimo Coração e do Coração Imaculado de Maria, peço-Vos a conversão dos pobres pecadores.
\end{flushleft}
\begin{center}
    Veni Creator \\ \textcolor{VioletRed3}{\scriptsize{(Oração para o dia 1 de Janeiro)}}
\end{center}
\begin{flushleft}
    Vinde, Espírito Criador, visitai as almas dos vossos fiéis; enchei de graça celestial os corações que Vós criastes.
    \vspace{.2cm} \\
    Vós, chamado o Consolador, dom do Deus Altíssimo, fonte viva, fogo, caridade, a unção espiritual.
    \vspace{.2cm} \\
    Vós, com Vossos sete dons, sois força da destra de Deus; Vós, o prometido pelo Pai, ditai-nos os gemidos da oração.
    \vspace{.2cm} \\
    Acendei a vossa luz em nossas almas, infundi o vosso amor em nossos corações; e a fraqueza da nossa carne, fortalecei-a com perpétua força.
    \vspace{.2cm} \\
    O inimigo, afugentai-o para longe; dai-nos quanto antes a paz; tendo-Vos por guia e condutor, venceremos todos os perigos.
    \vspace{.2cm} \\
    Por Vós conhecemos o Pai, e também o Filho; e que em Vós, Espírito de ambos, acreditamos em todo tempo.
    \vspace{.2cm} \\
    Glória a Deus Pai, ao Filho que ressuscitou e ao Espírito Santo Consolador. Pelos séculos dos séculos. Amém.
    \vspace{.2cm} \\
    \VbarRed{} Enviai o vosso Espírito e tudo será criado. \\
    \RbarRed{} E renovareis a face da terra.
    \vspace{.2cm} \\
    \textbf{\textit{Oremos:}} Ó Deus, que instruístes os corações dos vossos fiéis com a luz do Espírito Santo, concedei-nos amar, no mesmo Espírito, o que é reto, e gozar sempre a sua consolação. Por Cristo, Senhor Nosso. Amém.
\end{flushleft}
\newpage
\begin{center}
    Te Deum \\ \textcolor{VioletRed3}{\scriptsize{(Oração para o dia 31 de Dezembro)}}
\end{center}
\begin{flushleft}
    A Vós, ó Deus, louvamos; a Vós, Senhor, bendizemos \\
    A Vós, ó eterno Pai, adora toda a terra. \\
    A Vós, todos os Anjos, os Céus e todas as Potestades. \\
    A Vós, os Querubins e Serafins proclamam com incessante vozes: \\
    Santo, Santo, Santo, sois Vós, Senhor, Deus dos exércitos! \\
    Cheios estão os céus e a terra da majestade da vossa glória. \\
    A Vós, o glorioso coro dos Apóstolos. \\
    A Vós, o louvável número dos Profetas. \\
    A Vós, Vos louva o brilhante exército dos Mártires. \\
    A Vós confessa a Santa Igreja por toda a redondeza da terra. \\
    Pai de imensa majestade; \\
    Ao vosso adorável Filho, verdadeiro e único; \\
    E também ao Espírito Santo Consolador. \\
    Vós, ó Cristo, sois o Rei da glória. \\
    Vós sois o Filho eterno do Pai. \\
    Vós, para libertar o homem cuja carne havíeis de tomar, não rejeitastes o seio da Virgem. \\
    Vós, vencido o aguilhão da morte, abriste aos fiéis o Reino dos céus. \\
    Vós estais sentado à mão direita de Deus, na glória do Pai. \\
    Cremos que haveis de vir como Juiz. \\
    Por isso Vos rogamos: socorrei os vossos servos, que remistes com o vosso precioso Sangue. \\
    Permiti que sejamos do número dos vossos Santos na glória eterna. \\
    Salvai, Senhor, o vosso povo, e abençoai a vossa herança. \\
    Governai-os e exaltai-os eternamente. \\
    Todos os dias Vos bendizemos. \\
    E louvamos sempre o vosso Nome, por todos os séculos dos séculos. \\
    Dignai-Vos, Senhor, preservar-nos neste dia de todo o pecado. \\
    Tende piedade de nós, Senhor; tende piedade de nós. \\
    Faça-se, Senhor, a vossa misericórdia sobre nós, conforme esperamos em Vós. \\
    Em Vós, Senhor, esperei; não serei confundido eternamente.
    \vspace{.2cm} \\
    \VbarRed{} Bendito sois, Senhor, Deus de nossos pais. \\
    \RbarRed{} E dignos de louvor e glória pelos séculos.
    \vspace{.2cm} \\
    \VbarRed{} Bendigamos o Pai, o Filho e o Espírito Santo. \\
    \RbarRed{} Louvemo-lo e exaltemo-lo para sempre.
    \vspace{.2cm} \\
    \VbarRed{} Senhor, Vós sois bendito no firmamento dos céus. \\
    \RbarRed{} Sois dignos de louvor e glória para sempre.
    \newpage
    \VbarRed{} Bendiga minha alma ao Senhor. \\
    \RbarRed{} E nunca esqueça os seus muitos benefícios.
    \vspace{.2cm} \\
    \VbarRed{} Ouvi, Senhor, a minha oração. \\
    \RbarRed{} E chegue a Vós o meu clamor.
\end{flushleft}
\begin{center}
    Novena de Santa Teresa do Menino Jesus e da Sagrada Face
\end{center}
\begin{center}
    Sinal da Cruz
\end{center}
\begin{flushleft}
    Pelo Sinal, \grecrossRed{} da Santa Cruz, livrai-nos Deus, \grecrossRed{} Nosso Senhor, dos nossos \grecrossRed{} inimigos. Em nome do Pai, \grecrossRed{} e do Filho, e do Espírito Santo. Amém.
\end{flushleft}
\begin{center}
    Oração
\end{center}
\begin{flushleft}
    Santíssima Trindade: Pai, Filho e Espírito Santo: eu vos agradeço por todas as graças com que enriqueceste a vida de vossa serva, Santa Teresinha do Menino Jesus e da Sagrada Face, nestes 24 anos que passou na terra. E pelos méritos de tão querida santinha, concedei-me a graça que ardentemente vos peço: (\textcolor{VioletRed3}{fazer o pedido}), se for conforme a Vossa Santíssima Vontade e para a salvação de minha alma. Ajudai minha fé e minha esperança, Santa Teresinha, cumprindo mais uma vez vossa promessa de que ficareis no Céu a fazer o bem na terra, permitindo que eu ganhe um rosa em sinal de que alcançarei a graça pedida.
    \vspace{.2cm} \\
    Rezar 24 vezes, por cada ano de Santa Teresinha na terra:
\end{flushleft}
\begin{center}
    Glória \\ \textcolor{VioletRed3}{\scriptsize{(24 vezes)}}
\end{center}
\begin{flushleft}
    Glória ao Pai, e ao Filho e ao Espírito Santo. Assim como era no princípio, agora e sempre, por todos os séculos dos séculos. Amém.
    \vspace{.2cm} \\
    \VbarRed{} Santa Teresinha do Menino Jesus e da Sagrada Face. \\
    \RbarRed{} Rogai por mim.
\end{flushleft}
\newpage
\begin{center}
    Devoções dos Dias da Semana
\end{center}
\begin{itemize}
    \item Domingo: O Dia do Senhor
    \item Segunda-Feira: Almas do Purgatório
    \item Terça-Feira: Anjos da Guarda
    \item Quarta-Feira: São José
    \item Quinta-Feira: Eucaristia
    \item Sexta-feira: Paixão de Cristo
    \item Sábado: Nossa Senhora
\end{itemize}
\begin{center}
    Devoções dos Meses do Ano
\end{center}
\begin{itemize}
    \item Janeiro: Santíssimo Nome de Jesus
    \item Fevereiro: Sagrada Família
    \item Março: São José
    \item Abril: Eucaristia e Espírito Santo
    \item Maio: Virgem Maria
    \item Junho: Sagrado Coração de Jesus
    \item Julho: Preciosíssimo Sangue de Cristo
    \item Agosto: Vocações
    \item Setembro: Bíblia
    \item Outubro: Rosário
    \item Novembro: Fiéis Defuntos
    \item Dezembro: Natal
\end{itemize}
