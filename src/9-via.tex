% chktex-file 8
% chktex-file 19
% chktex-file 38
\newpage
\begin{center}
    \textbf{Via Crucis}
\end{center}
\begin{center}
    Sinal da Cruz
\end{center}
\begin{flushleft}
    Pelo Sinal, \grecrossRed{} da Santa Cruz, livrai-nos Deus, \grecrossRed{} Nosso Senhor, dos nossos \grecrossRed{} inimigos. Em nome do Pai, \grecrossRed{} e do Filho, e do Espírito Santo. Amém.
    \vspace{.2cm} \\
    \textit{Oração diante do altar:}
    \vspace{.2cm} \\
    Em união com Maria, a Mãe das dores, vamos, ó Jesus, percorrer o caminho doloroso por onde passastes para consumar a nossa redenção no Calvário. Oxalá esta meditação dos mistérios da vossa Paixão nos encha o coração de compunção por nossos pecados e de reconhecimento pelo vosso grande amor para conosco.
\end{flushleft}
\begin{center}
    I.\ Estação \\ Jesus é condenado à morte
\end{center}
\begin{flushleft}
    \VbarRed{} Nós Vos adoramos, ó Jesus e Vos bendizemos. \\
    \RbarRed{} Porque pela vossa Santa Cruz redimistes o mundo.
    \vspace{.2cm} \\
    Do Evangelho segundo São Mateus - (\textcolor{VioletRed2}{Mt 27, 22-23}).
    \vspace{.2cm} \\
    Pilatos perguntou: ``Que farei, então, com Jesus, que é chamado o Cristo?'' Todos responderam: ``Seja crucificado!'' Pilatos falou: ``Mas, que mal ele fez?'' Eles, porém, gritaram com mais força: ``Seja crucificado!''
    \vspace{.2cm} \\
    Pai Nosso, Ave Maria, Glória.
    \vspace{.2cm} \\
    \VbarRed{} Senhor, pequei. \\
    \RbarRed{} Tende piedade e misericórdia de mim.
\end{flushleft}
\begin{center}
    II.\ Estação \\ Jesus toma a sua Cruz
\end{center}
\begin{flushleft}
    \VbarRed{} Nós Vos adoramos, ó Jesus e Vos bendizemos. \\
    \RbarRed{} Porque pela vossa Santa Cruz redimistes o mundo.
    \vspace{.2cm} \\
    Do Evangelho segundo São Mateus - (\textcolor{VioletRed2}{Mt 16, 24-26}).
    \vspace{.2cm} \\
    Então Jesus disse aos discípulos: ``Se alguém quer vir após mim, negue-se a si mesmo, tome a sua cruz e siga-me, pois quem quiser salvar sua vida, a perderá; mas quem perder sua vida por causa de mim, a encontrará. Com efeito, que adianta a alguém ganhar o mundo inteiro, mais arruinar a sua vida? Que poderia dar em troca de sua vida?
    \vspace{.2cm} \\
    Pai Nosso, Ave Maria, Glória.
    \vspace{.2cm} \\
    \VbarRed{} Senhor, pequei. \\
    \RbarRed{} Tende piedade e misericórdia de mim.
\end{flushleft}
\begin{center}
    III.\ Estação \\ Jesus cai sob o peso da Cruz
\end{center}
\begin{flushleft}
    \VbarRed{} Nós Vos adoramos, ó Jesus e Vos bendizemos. \\
    \RbarRed{} Porque pela vossa Santa Cruz redimistes o mundo.
    \vspace{.2cm} \\
    Do Livro do Profeta Isaías - (\textcolor{VioletRed2}{Is 50, 5-7}).
    \vspace{.2cm} \\
    O Senhor Deus abriu-me os ouvidos, e eu não fui rebelde, nem recuei. Apresentei as costas aos que me feriam, e a face aos que me arrancavam a barba; não desviei o rosto dos insultos e dos escarros. O Senhor Deus é quem me ajuda: por isso, não fiquei envergonhado. Permaneço com o rosto impassível, duro como pedra, porque sei que não serei envergonhado.
    \vspace{.2cm} \\
    Pai Nosso, Ave Maria, Glória.
    \vspace{.2cm} \\
    \VbarRed{} Senhor, pequei. \\
    \RbarRed{} Tende piedade e misericórdia de mim.
\end{flushleft}\begin{center}
    IV.\ Estação \\ Jesus encontra Maria, sua Santíssima Mãe
\end{center}
\begin{flushleft}
    \VbarRed{} Nós Vos adoramos, ó Jesus e Vos bendizemos. \\
    \RbarRed{} Porque pela vossa Santa Cruz redimistes o mundo.
    \vspace{.2cm} \\
    Do Evangelho segundo São Lucas - (\textcolor{VioletRed2}{Lc 2, 34-35}).
    \vspace{.2cm} \\
    Simeão os abençoou e disse a Maria, sua Mãe: ``Este é destinado a ser causa de queda e de reerguimento de muitos em Israel, e a ser sinal de contradição. Assim serão revelados os pensamentos de muitos corações.\ Quanto a ti, uma espada te transpassará a alma''.
    \vspace{.2cm} \\
    Pai Nosso, Ave Maria, Glória.
    \vspace{.2cm} \\
    \VbarRed{} Senhor, pequei. \\
    \RbarRed{} Tende piedade e misericórdia de mim.
\end{flushleft}
\newpage
\begin{center}
    V.\ Estação \\ Simão Cirineu ajuda Jesus a carregar a Cruz
\end{center}
\begin{flushleft}
    \VbarRed{} Nós Vos adoramos, ó Jesus e Vos bendizemos. \\
    \RbarRed{} Porque pela vossa Santa Cruz redimistes o mundo.
    \vspace{.2cm} \\
    Do Evangelho segundo São Lucas - (\textcolor{VioletRed2}{Lc 23, 26}).
    \vspace{.2cm} \\
    Enquanto levavam Jesus, agarraram um certo Simão de Cirene, que voltava do campo, e puseram-lhe a cruz aos ombros, para que a carregasse atrás de Jesus.
    \vspace{.2cm} \\
    Pai Nosso, Ave Maria, Glória.
    \vspace{.2cm} \\
    \VbarRed{} Senhor, pequei. \\
    \RbarRed{} Tende piedade e misericórdia de mim.
\end{flushleft}
\begin{center}
    VI.\ Estação \\ Uma piedosa mulher enxuga a face de Jesus
\end{center}
\begin{flushleft}
    \VbarRed{} Nós Vos adoramos, ó Jesus e Vos bendizemos. \\
    \RbarRed{} Porque pela vossa Santa Cruz redimistes o mundo.
    \vspace{.2cm} \\
    Do Livro do Profeta Isaías - (\textcolor{VioletRed2}{Is 53, 1-3}).
    \vspace{.2cm} \\
    Quem acreditou naquilo que ouvimos, a quem foi revelado o braço do Senhor? Cresceu diante dele como um renovo, como raiz que nasce da terra seca: Não tinha aparência nem beleza para que o olhássemos, nem formosura que nos atraísse. Foi desprezado, como o último dos homens, homem de dores, experimentado no sofrimento, e quase escondíamos o rosto diante dele; desprezado, não lhe demos nenhuma importância.
    \vspace{.2cm} \\
    Pai Nosso, Ave Maria, Glória.
    \vspace{.2cm} \\
    \VbarRed{} Senhor, pequei. \\
    \RbarRed{} Tende piedade e misericórdia de mim.
\end{flushleft}
\newpage
\begin{center}
    VII.\ Estação\\ Jesus cai pela segunda vez
\end{center}
\begin{flushleft}
    \VbarRed{} Nós Vos adoramos, ó Jesus e Vos bendizemos. \\
    \RbarRed{} Porque pela vossa Santa Cruz redimistes o mundo.
    \vspace{.2cm} \\
    Do Livro do Profeta Isaías - (\textcolor{VioletRed2}{Is 53, 4-5}).
    \vspace{.2cm} \\
    Entretanto, ele assumiu as nossas fraquezas, e as nossas dores, ele as suportou. E nós achávamos que ele era um castigado, alguém por Deus ferido e humilhado. Mas ele foi ferido por causa de nossas iniquidades, esmagado por causa de nossos crimes. O castigo que nos dá a paz caiu sobre ele, por seus ferimentos fomos curados.
    \vspace{.2cm} \\
    Pai Nosso, Ave Maria, Glória.
    \vspace{.2cm} \\
    \VbarRed{} Senhor, pequei. \\
    \RbarRed{} Tende piedade e misericórdia de mim.
\end{flushleft}
\begin{center}
    VIII.\ Estação \\ Jesus consola as mulheres de Jerusalém
\end{center}
\begin{flushleft}
    \VbarRed{} Nós Vos adoramos, ó Jesus e Vos bendizemos. \\
    \RbarRed{} Porque pela vossa Santa Cruz redimistes o mundo.
    \vspace{.2cm} \\
    Do Evangelho segundo São Lucas - (\textcolor{VioletRed2}{Lc 23, 27-28}).
    \vspace{.2cm} \\
    Seguia-o uma grande multidão do povo, bem como mulheres, que batiam no peito e choravam por ele. Jesus, porém, voltou-se para elas e disse: ``Mulheres de Jerusalém, não choreis por mim! Chorai por vós mesmas e por vossos filhos!''
    \vspace{.2cm} \\
    Pai Nosso, Ave Maria, Glória.
    \vspace{.2cm} \\
    \VbarRed{} Senhor, pequei. \\
    \RbarRed{} Tende piedade e misericórdia de mim.
\end{flushleft}
\newpage
\begin{center}
    IX.\ Estação \\ Jesus cai pela terceira vez
\end{center}
\begin{flushleft}
    \VbarRed{} Nós Vos adoramos, ó Jesus e Vos bendizemos. \\
    \RbarRed{} Porque pela vossa Santa Cruz redimistes o mundo.
    \vspace{.2cm} \\
    Do Evangelho segundo São Mateus - (\textcolor{VioletRed2}{Mt 11, 28-30}).
    \vspace{.2cm} \\
    Vinde a mim, todos os que estais cansados e carregados de fardos, e eu vos darei descanso. Tomai sobre vós o meu jugo e aprendei de mim, porque sou manso e humilde de coração, e encontrais descanso para vós, pois o meu julgo é suave e o meu fardo é leve.
    \vspace{.2cm} \\
    Pai Nosso, Ave Maria, Glória.
    \vspace{.2cm} \\
    \VbarRed{} Senhor, pequei. \\
    \RbarRed{} Tende piedade e misericórdia de mim.
\end{flushleft}
\begin{center}
    X.\ Estação \\ Jesus é despojado de suas vestes
\end{center}
\begin{flushleft}
    \VbarRed{} Nós Vos adoramos, ó Jesus e Vos bendizemos. \\
    \RbarRed{} Porque pela vossa Santa Cruz redimistes o mundo.
    \vspace{.2cm} \\
    Do Evangelho segundo São Marcos - (\textcolor{VioletRed2}{Mc 15, 24}).
    \vspace{.2cm} \\
    Eles o crucificaram e repartiram suas vestes, tiram a sorte sobre elas, para ver que parte tocaria a cada um.
    \vspace{.2cm} \\
    Pai Nosso, Ave Maria, Glória.
    \vspace{.2cm} \\
    \VbarRed{} Senhor, pequei. \\
    \RbarRed{} Tende piedade e misericórdia de mim.
\end{flushleft}
\begin{center}
    XI.\ Estação \\ Jesus é pregado na Cruz
\end{center}
\begin{flushleft}
    \VbarRed{} Nós Vos adoramos, ó Jesus e Vos bendizemos. \\
    \RbarRed{} Porque pela vossa Santa Cruz redimistes o mundo.
    \vspace{.2cm} \\
    Do Evangelho segundo São Lucas - (\textcolor{VioletRed2}{Lc 23, 33-34}).
    \vspace{.2cm} \\
    Quando chegaram ao lugar chamado Calvário, ali crucificaram Jesus e os malfeitores: um à sua direita e outro à sua esquerda. Jesus dizia: ``Pai, perdoa-lhes! Eles não sabem o que fazem!'' Então repartiram suas vestes tirando a sorte.
    \vspace{.2cm} \\
    Pai Nosso, Ave Maria, Glória.
    \vspace{.2cm} \\
    \VbarRed{} Senhor, pequei. \\
    \RbarRed{} Tende piedade e misericórdia de mim.
\end{flushleft}
\begin{center}
    XII.\ Estação \\ Jesus morre na Cruz
\end{center}
\begin{flushleft}
    \VbarRed{} Nós Vos adoramos, ó Jesus e Vos bendizemos. \\
    \RbarRed{} Porque pela vossa Santa Cruz redimistes o mundo.
    \vspace{.2cm} \\
    Do Evangelho segundo São João - (\textcolor{VioletRed2}{Jo 19, 28-30}).
    \vspace{.2cm} \\
    Em seguida, sabendo Jesus que tudo estava consumado, para que se cumprisse a Escritura, disse: ``Tenho sede!'' Havia ali uma vasilha cheia de vinagre. Fixaram uma esponja embebida em vinagre num ramo de hissopo e a levaram à sua boca. Depois que tomou o vinagre, ele disse: ``Está consumado''. E, inclinando a cabeça, entregou o espírito.
    \vspace{.2cm} \\
    Pai Nosso, Ave Maria, Glória.
    \vspace{.2cm} \\
    \VbarRed{} Senhor, pequei. \\
    \RbarRed{} Tende piedade e misericórdia de mim.
\end{flushleft}
\begin{center}
    XIII.\ Estação \\ Jesus é descido da Cruz e entregue à sua Mãe
\end{center}
\begin{flushleft}
    \VbarRed{} Nós Vos adoramos, ó Jesus e Vos bendizemos. \\
    \RbarRed{} Porque pela vossa Santa Cruz redimistes o mundo.
    \vspace{.2cm} \\
    Do Evangelho segundo São João - (\textcolor{VioletRed2}{Jo 19, 38-40}).
    \vspace{.2cm} \\
    Depois disso, José de Arimateia, que era discípulo de Jesus, porém às escondidas por medo dos judeus, pediu a Pilatos permissão para retirar o corpo de Jesus. Pilatos o permitiu, e José foi e retirou o corpo. Veio também Nicodemos, aquele que anteriormente tinha ido a Jesus de noite. Ele trouxe uma mistura de mirra e aloés, cerca de cem libras. Eles pegaram o corpo de Jesus e o envolveram, com os perfumes, em faixas de linho, ao modo de como os judeus costumam sepultar.
    \vspace{.2cm} \\
    Pai Nosso, Ave Maria, Glória.
    \vspace{.2cm} \\
    \VbarRed{} Senhor, pequei. \\
    \RbarRed{} Tende piedade e misericórdia de mim.
\end{flushleft}
\newpage
\begin{center}
    XIV.\ Estação \\ Jesus é colocado no sepulcro
\end{center}
\begin{flushleft}
    \VbarRed{} Nós Vos adoramos, ó Jesus e Vos bendizemos. \\
    \RbarRed{} Porque pela vossa Santa Cruz redimistes o mundo.
    \vspace{.2cm} \\
    Do Evangelho segundo São João - (\textcolor{VioletRed2}{Jo 19, 40-42}).
    \vspace{.2cm} \\
    Eles pegaram o corpo de Jesus e o envolveram, com os perfumes, em faixas de linho, ao modo de como os judeus costumam sepultar. No lugar onde Jesus fora crucificado, havia um jardim e, no jardim, um túmulo novo, no qual ninguém ainda havia sido posto. Como era o dia de preparação dos judeus e o túmulo estava perto, ali puseram Jesus.
    \vspace{.2cm} \\
    Pai Nosso, Ave Maria, Glória.
    \vspace{.2cm} \\
    \VbarRed{} Senhor, pequei. \\
    \RbarRed{} Tende piedade e misericórdia de mim.
    \vspace{.2cm} \\
    \textit{No fim:}
    \vspace{.2cm} \\
    Jesus, morto por mim, concedei-me a graça de morrer num ato de perfeita caridade para convosco. Santa Maria, Mãe de Deus, rogai por mim, agora e na hora da minha morte. São José, meu Pai e Senhor, alcançai-me que morra com a morte dos justos. Amém.
    \vspace{.2cm} \\
    \textbf{\textit{Oremos:}} Ó Deus, Pai todo-poderoso, que em Jesus Cristo, teu Filho, assumiste as chagas e os sofrimentos da humanidade, hoje tenho a coragem de Te suplicar, como o ladrão arrependido: ``Lembra-Te de mim''. Estou aqui, sozinho na tua presença, na escuridão dessa prisão, pobre, nu, faminto e desprezado, e peço-Te para derramares sobre as minhas feridas o óleo do perdão e da consolação e o vinho duma fraternidade que fortalece o coração. Cura-me com tua graça e ensina-me a manter a esperança no meio do desespero. Continua, Pai misericordioso, a confiar em mim, a dar-me sempre uma nova oportunidade, a abraçar-me no teu amor infinito. Com a tua ajuda e o dom do Espírito Santo, também eu serei capaz de Te reconhecer e servir nos meus irmãos. Amém.
    \vspace{.2cm} \\
    Em nome do Pai, \grecrossRed{} e do Filho, e do Espírito Santo. Amém.
\end{flushleft}
