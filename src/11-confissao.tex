% chktex-file 8
% chktex-file 19
% chktex-file 38
\newpage
\begin{center}
    \textbf{Sacramento da Confissão}
\end{center}
\begin{center}
    Sinal da Cruz
\end{center}
\begin{flushleft}
    Pelo Sinal, \grecrossRed{} da Santa Cruz, livrai-nos Deus, \grecrossRed{} Nosso Senhor, dos nossos \grecrossRed{} inimigos. Em nome do Pai, \grecrossRed{} e do Filho, e do Espírito Santo. Amém.
\end{flushleft}
\begin{center}
    Vinde, Espírito Santo
\end{center}
\begin{flushleft}
    Vinde, Espírito Santo, enchei os corações dos vossos fiéis e acendei neles o fogo do vosso amor. \\
    \VbarRed{} Enviai o vosso Espírito e tudo será criado. \\
    \RbarRed{} E renovareis a face da terra.
    \vspace{.2cm} \\
    \textbf{\textit{Oremos:}} Ó Deus, que instruístes os corações dos vossos fiéis com a luz do Espírito Santo, concedei-nos amar, no mesmo Espírito, o que é reto, e gozar sempre a sua consolação. Por Cristo, Senhor Nosso. Amém.
    \vspace{.2cm} \\
    Meu Deus e Senhor, vou receber agora o santo Sacramento da Penitência. Ajudai-me para isso com o auxílio de vossa graça; porque nada posso sem vós. Enviai-me o Espírito Santo, para que conheça bem o número e a gravidade de meus pecados, devida e sinceramente deles me arrependa e faça um firme propósito de não pecar mais. Assisti-me com a vossa graça, para que confesse sinceramente os meus pecados e não cale nada do que deva dizer. Dai-me força para me emendar verdadeiramente. Amém.
    \vspace{.2cm} \\
    Santa Maria, Mãe de Deus, rogai por mim, pobre pecador, para que faça uma boa confissão e alcance o perdão de todos os meus pecados. Jesus, Maria, José, esclarecei-me, socorrei-me, salvai-me. Amém. Santo Anjo da Guarda e todos os Anjos e Santos de Deus, rogai por mim nesta hora. Amém.
    \vspace{.2cm} \\
    Pai Nosso, Ave Maria, Glória.
\end{flushleft}
\newpage
\begin{center}
    Exame de Consciência \\
    \hfill{} \break{}
    \textcolor{VioletRed3}{Os Dez Mandamentos da Lei de Deus} \\
    \hfill{} \break{}
    1\textordmasculine{} Mandamento
\end{center}
\begin{flushleft}
    Eu sou o Senhor, teu Deus, que te fez sair da terra do Egito, da casa da escravidão. Não terás outros deuses diante de mim. Não farás para ti imagem esculpida de nada que se assemelhe ao que existe lá em cima, nos céus, ou embaixo, na terra, ou nas águas que estão abaixo da terra. Não te prostrarás diante desses deuses e não os servirás. (Amar a Deus sobre todas as coisas). \\
\end{flushleft}
\begin{itemize}
    \item Deixei de rezar a oração da manhã e da noite por negligência? (pecado venial);
    \item Rezei orações sem devoção, rindo ou voluntariamente distraído? (pecado venial);
    \item Neguei alguma verdade da fé? (pecado mortal);
    \item Falei contra a Religião: zombei de coisas santas, p.ex., da Santa Missa? (pecado mortal ou venial, conforme a matéria e o modo);
    \item Mandei consultar espíritas, feiticeiros, benzendouros ou cartomantes? Fiz feitiços ou usei devoções supersticiosas? (pecado mortal, pode ser venial se for sem plena advertência ou conhecimento);
    \item Desconfiei de Deus, murmurando contra Ele, entregando-me ao desânimo nas desgraças e até ao desespero? (pecado venial muitas vezes; mortal se há desespero da misericórdia de Deus ou da salvação);
    \item Pequei e continuei a pecar contando com a misericórdia divina? (pecado mortal);
    \item Esperei não perder a vida, por milagre de Deus, expondo-me ao risco de vida ou a outro grave perigo? (pecado mortal; venial se não houve plena advertência).
\end{itemize}
\newpage
\begin{center}
    2\textordmasculine{} Mandamento
\end{center}
\begin{flushleft}
    Não pronunciarás o nome do Senhor, teu Deus, em vão.
\end{flushleft}
\begin{itemize}
    \item Disse o nome de Deus ou dos santos sem respeito? (pecado venial);
    \item Jurei falso? (se for verdadeiro juramento falso, é pecado mortal);
    \item Jurei à toa? (pecado venial);
    \item Roguei pragas, proferi maldições? (pecado mortal; venial, se for sem plena advertência, ou a matéria leve);
    \item Blasfemei, disse injúrias contra Deus ou os santos? (pecado mortal)
    \item Não cumpri as promessas que eu fiz a Deus e aos santos? (é grave, se a matéria o for e se teve a intenção de ficar gravemente obrigado. Caso contrário, é leve).
\end{itemize}
\begin{center}
    3\textordmasculine{} Mandamento
\end{center}
\begin{flushleft}
    Lembra-te de guardar o Dia do Senhor.
\end{flushleft}
\begin{itemize}
    \item Faltei por minha culpa à Missa toda ou grande parte dela nos domingos e festas de guarda? (pecado mortal; venial se a parte omitida não é das principais e for pequena);
    \item Trabalhei nos domingos e dias santos sem necessidade? - Diga que trabalho foi e quando tempo (pecado mortal, conforme o trabalho e o tempo).
\end{itemize}
\newpage
\begin{center}
    4\textordmasculine{} Mandamento
\end{center}
\begin{flushleft}
    Honrar teu pai e tua mãe, para que se prolonguem os teus dias na terra que o Senhor, teu Deus, te dá.
\end{flushleft}
\begin{itemize}
    \item Desobedeci, contrariei ou aborreci ou faltei com o respeito a meus superiores? (é geralmente leve; pode ser grave se a matéria é grave);
    \item Deixei de socorrê-los nas necessidades? - Diga em que foi (pecado mortal ou venial conforme a matéria e as circunstâncias);
    \item Desejei a meus pais grande (pecado mortal) ou pequeno mal (venial)?
\end{itemize}
\begin{flushleft}
    Examinem os pais a consciência a respeito de seus filhos se não houve descuido da sua educação e instrução religiosa; se lhes deram bons exemplos; se não deixaram faltar-lhes o necessário. \\
    Para aqueles que são patrões, a respeito dos funcionários: se os trataram bem, pagaram o devido salário, deram instruções e tempo para o cumprimento dos deveres religiosos. \\
    Os cidadãos a respeito do amor à Pátria: revolta contra a devida autoridade; pagamento do imposto e defesa do seu território.  \\
    Os eleitores examinem se voltaram ou não, e em quem votaram: se em pessoas competentes e de princípios cristãos, ou em outros só por serem amigos.
\end{flushleft}
\begin{center}
    5\textordmasculine{} Mandamento
\end{center}
\begin{flushleft}
    Não matarás.
\end{flushleft}
\begin{itemize}
    \item Briguei com meus irmãos ou outras pessoas? Injuriei, maltratei-os? (pecado venial; mortal quando há gravidade nestes maus tratos);
    \item Matei? Procurei matar? Desejei matar? (pecado grave) Provoquei aborto? (grave e excomunhão) Provoquei acidentes de trânsito? Fui imprudente ao guiar?;
    \item Feri alguém? (conforme o ferimento é grave ou leve);
    \item Tive raiva e inimizade. Desejo de vingança? (pecado mortal ou venial segundo o tempo longo ou curto e a matéria for leve);
    \item Fiz os outros pecarem por maus exemplos, conselhos, palavras ou obras? (pecado mortal ou venial segundo o pecado de que fomos causa).
\end{itemize}
\newpage
\begin{center}
    6\textordmasculine{} Mandamento
\end{center}
\begin{flushleft}
    Não pecarás contra a castidade.
\end{flushleft}
\begin{center}
    9\textordmasculine{} Mandamento
\end{center}
\begin{flushleft}
    Não desejarás a mulher do próximo. \\
    \hfill{} \break{}
    Todos os pecados de luxúria ou impureza são mortais, exceto se não há advertência e pleno consentimento. Os pecados, porém, de falta de pudor ou imodéstia podem ser graves ou leves: depende isto do maior ou menor perigo de impureza, do escândalo que causa e da intenção da pessoa.
\end{flushleft}
\begin{itemize}
    \item Pensei involuntariamente em coisas desonestas?;
    \item Quis ver, ouvir, falar, ler ou fazer coisas desonestas? (dizer com que classe; homem ou mulher, perante ou não, etc, sem citar nomes);
    \item Olhei com prazer, fale, ouvi coisas desonestas?;
    \item Vi figuras ou li escritos imorais? Decotes exagerados? Faltei à modéstia ao vestir-me ou despir-me ou em outras ocasiões? - Fiz ações desonestas ou deixei que as fizessem em mim? Com parentes, com pessoas do mesmo ou de outro sexo? (dizer se é casado ou não, e se a pessoa com que se fez o é ou não);
    \item Evitei filhos no matrimônio? Tomei remédios ou coisas semelhantes para tal fim?
    \item Ensinei. Provoquei ou ajudei a outros por palavras, ações, etc., a cometerem o pecado desonesto?;
    \item Expus-me a ocasiões próximas de pecado, como são certas pessoas ou companhias, teatros e festas indecentes, lugares suspeitos, livros obscenos?;
    \item Os noivos examinem-se sobre se foram castos com suas noivas. Os jovens, sobre as festas, as danças, filmes indecentes, os livros obscenos.
\end{itemize}
\newpage
\begin{center}
    7\textordmasculine{} Mandamento
\end{center}
\begin{flushleft}
    Não roubarás.
\end{flushleft}
\begin{center}
    10\textordmasculine{} Mandamento
\end{center}
\begin{flushleft}
    Não cobiçarás as coisas alheias. \\
    \hfill{} \break{}
    Pecados graves ou leves: segundo os objetos injustamente tirados ou segundo as circunstâncias.
\end{flushleft}
\begin{itemize}
    \item Furtei? Desejei furtar alguma coisa?;
    \item Aceitei, comprei ou tive coisas roubadas? Fiquei com coisas emprestadas ou achadas sem procurar o dono?;
    \item Não paguei ou demorei em pagar as dívidas?;
    \item Prejudiquei aos outros nos seus bens, entregando-lhes objetos, enganando-os no preço ou na medida do que lhes vendi?;
    \item Ajudei outros a fazerem tais pecados por ordem, conselhos ou silêncio culposo? Gastei no jogo? Não sustentei a família?.
\end{itemize}
\begin{center}
    8\textordmasculine{} Mandamento
\end{center}
\begin{flushleft}
    Não apresentarás um falso testamento contra teu próximo.
\end{flushleft}
\begin{itemize}
    \item Menti? Menti com prejuízo para os outros? (geralmente, pecado venial: mortal, se o prejuízo causado for grave);
    \item Murmurei da vida alheia, critiquei ou revelei faltas do próximo? (pecado mortal se a murmuração ou falta descoberta for grave);
    \item Pensei mal do próximo sem razão? (pecado leve ou grave, conforme o que se pensou);
    \item Caluniei? Provoquei ou favoreci críticas ao próximo? (pecado mortal ou venial conforme a falta inventada ou apontada for grave ou leve);
    \item Causei discórdia por minhas murmurações? (pecado mortal ou venial conforme as discórdias causadas).
\end{itemize}
\newpage
\begin{center}
    \textcolor{VioletRed3}{Os Dez Mandamentos da Lei de Deus} \\
    \hfill{} \break{}
    1\textordmasculine{} Mandamento da Igreja
\end{center}
\begin{flushleft}
    Participar da Missa inteira nos domingos e em outras festas de guarda e abster-se de ocupações de trabalho. \\
    \hfill{} \break{}
    Dias de Preceitos da Igreja:
    \begin{itemize}
        \item  Santa Mãe de Deus, Maria;
        \item Epifania do Senhor;
        \item Ascensão do Senhor;
        \item Corpus Christi;
        \item Santos Apóstolos São Pedro e São Paulo;
        \item Assunção de Nossa Senhora;
        \item Todos os Santos;
        \item Imaculada Conceição;
        \item Natal do Senhor.
    \end{itemize}
\end{flushleft}
\begin{center}
    2\textordmasculine{} Mandamento da Igreja
\end{center}
\begin{flushleft}
    Confessar-se ao menos uma vez por ano.
\end{flushleft}
\begin{center}
    3\textordmasculine{} Mandamento da Igreja
\end{center}
\begin{flushleft}
    Receber o sacramento da Eucaristia ao menos pela Páscoa da Ressurreição.
\end{flushleft}
\begin{center}
    4\textordmasculine{} Mandamento da Igreja
\end{center}
\begin{flushleft}
    Jejuar e abster-se de carne, conforme manda a Santa Mãe Igreja. \\
    \hfill{} \break{}
    Dias de jejum e abstinência obrigatórios: \\
    Quarta-Feira de Cinzas e Sexta-Feira Santa.
\end{flushleft}
\begin{center}
    5\textordmasculine{} Mandamento da Igreja
\end{center}
\begin{flushleft}
    Ajudar a Igreja em suas necessidades.
\end{flushleft}
\newpage
\begin{center}
    Textos para Meditação
\end{center}
\begin{flushleft}
    Evangelho segundo São Mateus - \textcolor{VioletRed3}{Mt 5-7}; \\
    Carta de São Paulo aos Romanos - \textcolor{VioletRed3}{Rm 12-15}; \\
    Primeira Carta de São Paulo aos Coríntios - \textcolor{VioletRed3}{1Cor 12-13}; \\
    Carta de São Paulo aos Gálatas - \textcolor{VioletRed3}{Gl 5}; \\
    Carta de São Paulo aos Efésios - \textcolor{VioletRed3}{Ef 4-5}.
\end{flushleft}
\begin{center}
    Ato de Contrição
\end{center}
\begin{flushleft}
    Senhor meu Jesus Cristo, Deus e homem verdadeiro, Criador e Redentor meu, por serdes Vós quem sois, sumamente bom e digno de ser amado sobre todas as coisas, e porque Vos amo e estimo, pesa-me, também, de ter perdido o Céu e merecido o Inferno; e proponho-me firmemente, ajudado com o auxílio da vossa divina graça, emendar-me e nunca mais Vos tornar a ofender. Espero alcançar o perdão de minhas culpas pela vossa infinita misericórdia. Amém.
\end{flushleft}
\begin{center}
    Oração ao Sagrado Coração de Jesus
\end{center}
\begin{flushleft}
    Ó Deus, que no coração de vosso Filho, ferido por nossos pecados, Vos dignais prodigalizar-nos os infinitos tesouros de Amor, fazei, Vos rogamos, que rendendo-Lhe o preito de nossa devoção e piedade, também cumpramos dignamente para com Ele o dever de reparação. Pelo mesmo Jesus Cristo, Senhor Nosso. Amém.
\end{flushleft}
\begin{center}
    Ato de reparação
\end{center}
\begin{flushleft}
    Dulcíssimo Jesus, cuja infinita caridade para com os homens é por eles tão ingratamente correspondida com esquecimentos, friezas e desprezos, eis-nos aqui prostrados na Vossa presença, para Vos desagravarmos, com especiais homenagens, da insensibilidade tão insensata e das nefandas injúrias com que é de toda parte alvejado o Vosso amorosíssimo coração.
    \vspace{.2cm} \\
    Reconhecendo, porém, com a mais profunda dor, que também nós mais de uma vez cometemos as mesmas indignidades, para nós, em primeiro lugar, imploramos a Vossa misericórdia, prontos a expiar não só as próprias culpas, senão também as daqueles que, errando longe do caminho da salvação, ou se obstinam na sua infidelidade, não Vos querendo como pastor e guia, ou, conculcando as promessas do batismo, sacudiram o suavíssimo jugo da Vossa santa lei.
    \newpage
    De todos estes tão deploráveis crimes, Senhor, queremos nós hoje desagravar-Vos, mais particularmente da licença dos costumes e imodéstia do vestido, de tantos laços de corrupção armados à inocência, da violação dos dias santificados, das execrandas blasfêmias contra Vós e Vossos Santos, dos insultos ao Vosso Vigário e a todo o Vosso clero, do desprezo e das horrendas e sacrílegas profanações do Sacramento do divino amor e, enfim, dos atentados e rebeldias das nações contra os direitos e o Magistério da Vossa Igreja.
    \vspace{.2cm} \\
    Oh! Se pudéssemos lavar com o próprio sangue tantas iniquidades!
    \vspace{.2cm} \\
    Entretanto, para reparar a honra divina ultrajada, Vos oferecemos, juntamente com os merecimentos da Virgem Mãe, de todos os santos e almas piedosas, aquela infinita satisfação, que Vós oferecestes ao eterno Pai sobre a cruz, e que não cessais de renovar todos os dias sobre nossos altares.
    \vspace{.2cm} \\
    Ajudai-nos Senhor, com o auxílio da Vossa graça, para que possamos, como é nosso firme propósito, com a vivência da fé, com a pureza dos costumes, com a fiel observância da lei e caridade evangélicas, reparar todos os pecados cometidos por nós e por nosso próximo, impedir, por todos os meios, novas injúrias de Vossa divina Majestade e atrair ao Vosso serviço o maior número de almas possível.
    \vspace{.2cm} \\
    Recebei, ó benigníssimo Jesus, pelas mãos de Maria santíssima reparadora, a espontânea homenagem deste nosso desagravo, e concedei-nos a grande graça de perseverarmos constantes, até à morte, no fiel cumprimento de nossos deveres e no Vosso santo serviço, para que possamos chegar todos à pátria bem-aventurada, onde Vós com o Pai e o Espírito Santo viveis e reinais por todos os séculos dos séculos. Amém.
\end{flushleft}
\begin{center}
    Jaculatórias
\end{center}
\begin{flushleft}
    \VbarRed{} Jesus, manso e humilde de coração. \\
    \RbarRed{} Fazei o nosso coração semelhante ao vosso. \\
    \VbarRed{} Coração sacratíssimo de Jesus. \\
    \RbarRed{} Tende piedade de nós. \\
    \VbarRed{} Coração sacratíssimo e misericordioso de Jesus. \\
    \RbarRed{} Dai-nos a paz.
\end{flushleft}
\newpage
\begin{center}
    Meditação: As Quatro Considerações \\
    \hfill{} \break{}
    \textcolor{VioletRed3}{Primeira Consideração: O Túmulo}
\end{center}
\begin{flushleft}
    Transporta-te em espírito à beira dum túmulo. Imagina-te uma cova. Que é que lá verias? Um cadáver apodrecido, roído por milhares de bichos, tão feio, espalhando cheiro, tão desagradável, que te custaria, suportar tal presença. Eis aqui o homem, rei da terra, a criatura mais formosa e mais nobre deste mundo; um montão de ossos, uma comida de vermes!
    \vspace{.2cm} \\
    E que foi que o reduziu a estado tão horrível? Foi a morte. E quem veio introduzir a morte neste mundo? O pecado. Adão e Eva comeram o fruto proibido, desobedeceram a Deus e foram condenados à morte; e com eles, todo o gênero humano. Eis a consequência dum só pecado mortal! É o pecado que transforma o corpo humano, obra tão esplêndida e artificiosa da onipotência divina, num monturo de podridão. Foi um único pecado mortal, que, num momento, transformou os anjos mais formosos no estado mais feio e mais abominável que há: em demônios. Que grande mal, pois, deve ser o pecado mortal! Sim, muito mais feio e mais horrível do que um cadáver reduzido a podridão.
    \vspace{.2cm} \\
    Contam que, na Antiguidade, um tirano, para atormentar o seu inimigo do modo mais cruel possível, mandou-o ligar vivo a um cadáver. Que castigo horrível dia e noite ser amarrado a um cadáver apodrecido! Muito mais feia, porém, é a alma manchada com o pecado mortal. Essa alma, feita à imagem de Deus, outrora tão formosa, um templo de Deus, mais bela que o mais lindo jardim de flores, agora uma morada, uma escrava do demônio! E, talvez, já haja muito tempo que vendeste, entregaste tua alma ao demônio, dando entrada em teu coração a este teu inimigo capital, pelo pecado grave. Ah! Como és infeliz agora! Em lugar de paz e de alegria, agora remorsos horríveis.
    \vspace{.2cm} \\
    Não permitais, meu Deus, que mais uma vez entregue minha alma imortal ao demônio, dando a meu corpo um prazer proibido, satisfazendo os desejos da carne, que sempre se revolta contra o meu espírito; desta carne que um dia vai ser reduzida a pó e cinza, a uma comida de bichos. Fazei que vença as minhas más inclinações, principalmente esta \textcolor{VioletRed3}{(intenção)} a fim de que um dia meu corpo ressuscite glorioso, para participar da glória celeste.
\end{flushleft}
\newpage
\begin{center}
    \textcolor{VioletRed3}{Segunda Consideração: O Céu}
\end{center}
\begin{flushleft}
    Ergue teu Espírito ao céu. Imagina a coisa mais bela, mais sublime que há neste mundo. Talvez um mimoso jardim no brilho das mais belas flores; ou uma cidade, segundo a descreve S. João no seu Apocalipse, uma cidade com ruas de ouro puro, com portas de pérolas brilhantes, com muros de pedras preciosas. Tudo isso, comparado ao céu, não é mais nada do que a fraca luz duma pequena lâmpada, em comparação ao sol radioso.
    \vspace{.2cm} \\
    Imagina um homem, a quem foi concedido gozar, talvez uma vida inteira, todas as alegrias e prazeres que desde Adão pôde experimentar um pobre mortal. Todos esses gozos e deleites, comparados à Glória do céu, são como uma gota d'água em comparação ao oceano imenso. E este lugar de delícias era destinado para ti; mas, pelo pecado mortal, perdeste o direito de entrar naquela mansão celeste. Um único pecado mortal e perdido está o céu com suas delícias.
    \vspace{.2cm} \\
    Ó meu Deus, que coisa horrível deve ser o pecado mortal, que nos priva de um bem tão grande e sublime! E tu o soubeste, minha alma; e, apesar disto, cometeste o pecado mortal, renunciaste à tua eterna salvação no céu, talvez por algum dinheiro, que tiraste, talvez por outras tantas injustiças que cometeste contra o próximo, talvez por um prazer ilícito, por um pecado desonesto! Ah, que loucura, vender, perder sua eterna felicidade por um momento de prazeres proibidos; trocar o céu por algumas moedas, um punhado, de bens e riquezas passageiras!
    \vspace{.2cm} \\
    Porventura queres de novo perder o céu, cometendo um pecado mortal? Transporta o teu espírito mais uma vez àquela região celeste e contempla o Senhor do céu sentado no seu trono cercado de majestade tremenda, de glória indizível. Imagina os milhares e milhares de espíritos angélicos, trêmulos, prostrados diante do trono de Deus que, com as faces veladas e cheios de respeito e santa reverência cantam incessantemente o ``Santo, Santo, Santo''. Eis, como o céu e a terra se dobram diante da majestade divina --- e como tudo obedece à santíssima vontade de Deus.
    \newpage
    E tu, criatura tão vil e miserável, te atreveste a negar obediência a este Deus tão santo, tão forte, tão poderoso e, ao mesmo tempo, tão bondoso calcando aos pés a sua lei, transgredindo os seus mandamentos, provocando e desprezando a sua justa ira, afligindo amargamente o seu Coração paterno, que tanto te ama e te encheu de tantos benefícios! Prostra-te de joelhos na santíssima presença de Deus e, do fundo de teu coração, dize-lhe: Ó Deus de santidade e de misericórdia infinita, diante de quem o céu e a terra se inclinam, detesto agora sinceramente todos os pecados, com que na minha maldade ofendi a vossa divina majestade, desprezando a bondade do melhor dos pais. Ah, lançai um olhar de compaixão sobre mim, vosso filho ingrato, pois prometo ser agora e sempre um filho obediente, e não tornar a ofender-vos pelo pecado mortal.
\end{flushleft}
\begin{center}
    \textcolor{VioletRed3}{Terceira Consideração: O Inferno}
\end{center}
\begin{flushleft}
    Desce em espírito ao inferno --- e contempla os tormentos horríveis, que lá sofrem os condenados. --- Queimar um pouco o dedo já causa uma grande dor --- por nada deste mundo o deixarias, por uma hora inteira, no fogo. --- Os condenados, porém, sofrem num fogo muito mais ardente do que o nosso, tormentos horríveis. Imagina o rico avarento sepultado e enterrado no meio das chamas, e nem uma gota d'água lhe é concedida para refrescar a sua língua. E quanto tempo são atormentados os condenados? Talvez um dia, um ano, ou um século Isto não seria nada. Judas lá está no inferno há quase dois mil anos sem um sossego ou descanso e sofrerá mais do que estes dois mil, nunca será um pouco aliviado, jamais virá o fim de tantos tormentos e dores, pois as penas são eternas, nunca terão fim.
    \vspace{.2cm} \\
    Os condenados são sempre atormentados por dois pensamentos: ``nunca'' e ``sempre''. Nunca sair, sempre ficar na companhia das criaturas mais perversas e desgraçadas. Agora, pergunta-te, minha alma, qual a causa de tormentos tão horríveis? É o pecado o pecado mortal. E basta um só pecado mortal para te fazer merecedor de tais castigos da justiça divina. Se o bom Deus, o Deus de misericórdia, que não quer a morte do pecador, mas sim que se converta e viva, se Ele castiga tão severamente o pecado, por acaso não deves estremecer e temer de tornar a ofender pelo pecado mortal este juiz tremendo e severo? Ah, promete agora emendar-te. Ainda há tempo. Sim, meu Deus, antes morrer, que vos ofender pelo pecado mortal. Dai-me força para evitá-lo, principalmente este \textcolor{VioletRed3}{(Aqui diga o pecado a que te vês mais inclinado)}.
\end{flushleft}
\newpage
\begin{center}
    \textcolor{VioletRed3}{Quarta Consideração: O Calvário}
\end{center}
\begin{flushleft}
    Transporta teu espírito ao Calvário e contempla a Jesus Crucificado. As dores que teu Salvador sofre são tão horríveis que te deviam mover à compaixão mesmo se fosse o teu inimigo mortal, que lá padecesse. Mas é teu Salvador! Contempla-o: desde a ponta dos pés até a cabeça não há nem um ponto do seu corpo que não fosse martirizado; todo o corpo ferido, todo o corpo uma chaga. A cabeça é atormentada pela coroa de espinhos agudos; a boca pela sede ardente; as mãos e os pés são transpassados por pregos duros; a alma santíssima é abismada no mais profundo abandono.
    \vspace{.2cm} \\
    O verme, quando pisado, pode torcer-se. Jesus nem sequer pode mover-se na Cruz. E quem é aquele que tão cruelmente é maltratado? Talvez um malfeitor? Não. É o mais santo, o mais inocente é o próprio Filho de Deus. E por que se deixou pregar na Cruz? Por causa de teus pecados, para salvar-te da morte eterna do inferno, para reconciliar-te com Deus: Jesus, o Filho de Deus, morre na Cruz. Para salvar o servo, o Filho é condenado à morte! Ó, que amor! E tu soubeste quantas dores, quantos açoites, quantas gotas de seu sangue preciosíssimo custou a teu Jesus tua alma imortal!
    \vspace{.2cm} \\
    E, apesar disto, calcaste o sangue de Deus aos pés, cometendo o pecado mortal. Sim, para pagar a tua desobediência e o teu orgulho, Jesus carregou a sua Cruz. Para satisfazer por teus pensamentos e desejos pecaminosos foram-lhe enterrados aqueles espinhos pungentíssimos. Aquela sede ardente sofreu Jesus por causa de tantas palavras livres e indecentes ou ofensivas ao próximo; e, para pagar tantas ações ilícitas e pecaminosas, Jesus recebeu aqueles açoites horríveis e se deixou cravar com pregos duros no lenho da cruz.
    \vspace{.2cm} \\
    Minha alma, queres pregar mais uma vez a teu Jesus na cruz? Ajoelha-te diante da imagem de teu Jesus crucificado; pede-lhe perdão por tanta ingratidão e promete nunca mais ofender a teu amável Salvador. Dize-lhe de coração contrito: ``Ah, meu Jesus, por amor de vossas cinco chagas, por amor de vosso sangue preciosíssimo, lavai a minha pobre alma de toda a mancha do pecado. Deixai cair sobre minha alma uma só gota de vosso precioso sangue, tão copiosamente derramado, e minha alma será inteiramente purificada. E poderei chamar-me outra vez vosso filho. Ó doce Jesus, que tanto me amais: Fazei que eu vos ame cada vez mais!''
\end{flushleft}
