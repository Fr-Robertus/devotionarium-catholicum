% chktex-file 8
%chktex-file 19
\newpage
\begin{center}
    \textbf{Sumário da Doutrina}
\end{center}
\begin{center}
    Os artigos de Fé
\end{center}
\begin{itemize}
    \item Creio em Deus Pai todo-poderoso.
    \item E em Jesus Cristo, seu Filho único, Nosso Senhor.
    \item Jesus Cristo foi concebido pelo poder do Espírito Santo, nasceu da Virgem Maria.
    \item Jesus Cristo padeceu sob Pôncio Pilatos, foi crucificado, morto e sepultado.
    \item Jesus Cristo desceu aos Infernos, ressuscitou dos mortos no terceiro dia.
    \item Jesus subiu aos céus, está sentado à direita de Deus Pai todo-poderoso.
    \item Donde virá julgar os vivos e os mortos.
    \item Creio no Espírito Santo.
    \item Creio na Igreja Católica.
    \item Creio no perdão dos Pecados.
    \item Creio na ressurreição da carne.
    \item Creio na Vida eterna.
\end{itemize}
\begin{center}
    Os Dez Mandamentos
\end{center}
\begin{enumerate}
    \item Eu sou o Senhor, teu Deus, que te fez sair da terra do Egito, da casa da escravidão. Não terás outros deuses diante de mim. Não farás para ti imagem esculpida de nada que se assemelhe ao que existe lá em cima, nos céus, ou embaixo, na terra, ou nas águas que estão abaixo da terra. Não te prostrarás diante desses deuses e não os servirás.
    \item Não pronunciarás o nome do Senhor, teu Deus, em vão.
    \item Lembra-te de guardar o Dia do Senhor.
    \item Honrar teu pai e tua mãe, para que se prolonguem os teus dias na terra que o Senhor, teu Deus, te dá.
    \item Não matarás.
    \item Não pecarás contra a castidade.
    \item Não roubarás.
    \item Não apresentarás um falso testamento contra teu próximo.
    \item Não desejarás a mulher do próximo.
    \item Não cobiçarás as coisas alheias.
\end{enumerate}
\begin{center}
    Os Cinco Mandamentos da Igreja
\end{center}
\begin{itemize}
    \item Participar da Missa inteira nos domingos e em outras festas de guarda e abster-se de ocupações de trabalho.
    \item Confessar-se ao menos uma vez por ano.
    \item Receber o sacramento da Eucaristia ao menos pela Páscoa da Ressurreição.
    \item Jejuar e abster-se de carne, conforme manda a Santa Mãe Igreja.
    \item Ajudar a Igreja em suas necessidades.
\end{itemize}
\begin{center}
    Lei do jejum e abstinência
\end{center}
\begin{itemize}
    \item Toda Sexta-Feira do ano é dia de penitência, a não ser que coincida com alguma solenidade do calendário litúrgico. Nesse dia os fiéis devem abster-se de comer carne ou outro alimento, ou praticar alguma forma de penitência, principalmente alguma obra de caridade ou algum exercício de piedade.
    \item A Quarta-Feira de Cinzas e a Sexta-Feira Santa, memória da Paixão e Morte de Cristo, são dias de jejum e abstinência. A abstinência pode ser substituída pelos próprios fiéis por outra prática de penitência, caridade ou piedade, particularmente pela participação nesses dias na Sagrada Liturgia.
    \item Idade de obrigação: a abstinência obriga a partir dos 14 anos completos; o jejum a partir dos 18 anos completos até os 60 anos começados.
\end{itemize}
\newpage
\begin{center}
    Os Doze Apóstolos
\end{center}
\begin{enumerate}
    \item Pedro (Simão);
    \item Bartolomeu;
    \item André;
    \item Filipe;
    \item Tomé;
    \item Tiago;
    \item João;
    \item Tiago (O menor);
    \item Judas Tadeu;
    \item Judas Iscariotes - Matias;
    \item Simão (O Zelotes);
    \item Mateus.
\end{enumerate}
\begin{center}
    Os Sete Sacramentos
\end{center}
\begin{enumerate}
    \item Batismo;
    \item Confirmação ou Crisma;
    \item Penitência;
    \item Eucaristia;
    \item Unção dos Enfermos;
    \item Ordem;
    \item Matrimônio.
\end{enumerate}
\begin{center}
    As Três Virtudes Teologais
\end{center}
\begin{itemize}
    \item Fé;
    \item Esperança;
    \item Caridade.
\end{itemize}
\newpage
\begin{center}
    As Quatro Virtudes Cardeais
\end{center}
\begin{itemize}
    \item Prudência;
    \item Justiça;
    \item Fortaleza;
    \item Temperança.
\end{itemize}
\begin{center}
    Os Sete Dons do Espírito Santo
\end{center}
\begin{enumerate}
    \item Sabedoria;
    \item Inteligência;
    \item Conselho;
    \item Fortaleza;
    \item Ciência;
    \item Piedade;
    \item Temor de Deus.
\end{enumerate}
\begin{center}
    Sete Obras de Misericórdia Espirituais
\end{center}
\begin{itemize}
    \item Dar bom conselho;
    \item Ensinar os ignorantes;
    \item Corrigir os que erram;
    \item Consolar os aflitos;
    \item Perdoar as injúrias;
    \item Sofrer com paciência as fraquezas do próximo;
    \item Rogar a Deus pelos vivos e defuntos.
\end{itemize}
\newpage
\begin{center}
    Os Doze Frutos do Espírito Santo
\end{center}
\begin{enumerate}
    \item Caridade;
    \item Paz;
    \item Benignidade;
    \item Longanimidade;
    \item Fidelidade;
    \item Continência;
    \item Alegria;
    \item Paciência;
    \item Bondade;
    \item Mansidão;
    \item Modéstia;
    \item Castidade.
\end{enumerate}
\begin{center}
    Sete Obras de Misericórdia Corporais
\end{center}
\begin{enumerate}
    \item Dar de comer a quem tem fome;
    \item Dar de beber a quem tem sede;
    \item Vestir os nus;
    \item Dar pousada aos peregrinos;
    \item Visitar os enfermos e encarcerados;
    \item Remir os cativos;
    \item Enterrar os mortos.
\end{enumerate}
\newpage
\begin{center}
    As Oito Bem-Aventuranças
\end{center}
\begin{itemize}
    \item Bem-aventurados os pobres em espírito, porque deles é o Reino de Deus.
    \item Bem-aventurados os mansos, porque eles possuirão a terra.
    \item Bem-aventurados os que aflitos, porque serão consolados.
    \item Bem-aventurados os que têm fome e sede de justiça, poque serão saciados.
    \item Bem-aventurados os misericordiosos, porque alcançarão misericórdia.
    \item Bem-aventurados os puros de coração, porque verão a Deus.
    \item Bem-aventurados os que promovem a paz, porque serão chamados filhos de Deus.
    \item Bem-aventurados os que são perseguidos por causa da justiça, porque deles é o Reino dos Céus.
\end{itemize}
\begin{center}
    Os Sete Pecados Capitais e as Virtudes opostas
\end{center}
\begin{enumerate}
    \item Orgulho - Humildade
    \item Avareza - Generosidade
    \item Inveja - Amor ao próximo
    \item Ira - Mansidão
    \item Luxúria - Castidade
    \item Gula - Temperança
    \item Preguiça - Diligência
\end{enumerate}
\begin{center}
    Seis pecados contra o Espírito Santo
\end{center}
\begin{enumerate}
    \item Desesperar da salvação;
    \item Presunção de se salvar sem merecimento;
    \item Contradizer a verdade conhecida por tal;
    \item Ter inveja das mercês que Deus faz a outros;
    \item Obstinação no pecado;
    \item Impenitência final.
\end{enumerate}
\newpage
\begin{center}
    Quatro pecados que bradam ao Céu
\end{center}
\begin{itemize}
    \item Homicídio voluntário;
    \item Pecado sensual contra a natureza;
    \item Opressão dos pobres;
    \item Não pagar a quem trabalha.
\end{itemize}
\begin{center}
    Cooperação e cumplicidade com os pecados alheios
\end{center}
\begin{itemize}
    \item Participando neles direta ou voluntariamente;
    \item Mandando, aconselhando, louvando ou aprovando esses pecados;
    \item Não os revelando ou não os impedindo, quando a isso somos obrigados;
    \item Protegendo os que fazem o mal.
\end{itemize}
\begin{center}
    Os três principais gêneros de boas obras
\end{center}
\begin{itemize}
    \item Oração
    \item Jejum
    \item Esmola
\end{itemize}
\begin{center}
    Conselhos Evangélicos
\end{center}
\begin{itemize}
    \item Pobreza voluntária
    \item Castidade
    \item Obediência
\end{itemize}
\begin{center}
    Os novíssimos
\end{center}
\begin{itemize}
    \item Morte;
    \item Juízo;
    \item Inferno;
    \item Paraíso.
\end{itemize}
\begin{center}
    Dogmas Marianos
\end{center}
\begin{enumerate}
    \item Maternidade divina;
    \item Virgindade perpétua;
    \item Imaculada Conceição;
    \item Assunção da Virgem Maria.
\end{enumerate}
